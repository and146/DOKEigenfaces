\documentclass{beamer}
\mode<presentation>{
	\usetheme{Warsaw}
	\usecolortheme{beetle}
	\setbeamercovered{transparent}
}
\usepackage[utf8]{inputenc}
\usepackage[czech]{babel}
%\usepackage{a4wide}
\usepackage{graphicx}
\usepackage{courier}
\usepackage{amsmath}
\usepackage{amsfonts}

\usepackage{ucs}
\usepackage{palatino}

\author{\textbf{David Andrešič}}
\title{\textbf{Dokumentografické informační systémy} \\ Projekt: \textit{Rozpoznání člověka z obrazu pomocí metody PCA}}
\institute[ ]{VŠB - Technická univerzita Ostrava}
\date{\today}

\begin{document}

%%%%%%%%%%%%%%%%%%%%% TITULKA %%%%%%%%%%%%%%%%

\begin{frame}
  \titlepage
\end{frame}

%%%%%%%%%%%%%%%%%%%%% OSNOVA %%%%%%%%%%%%%%%%

\begin{frame}
  \frametitle{Obsah}
  %\tableofcontents[pausesections] <-- pausesections = na klepnuti
  \tiny  
  \tableofcontents
\end{frame}

%%%%%%%%%%%%%%%%%%%%% TEORET. UVOD %%%%%%%%%%%%%%%%

\section{Motivace}
\begin{frame}

\begin{center}
\frametitle{Motivace a problémy}
	\huge Motivace a problémy\\[1.5cm]
	\large Motivace
\end{center}

\end{frame}

\begin{frame}
\frametitle{Motivace}
... aneb \textit{"NSA a ti další"} ...
\begin{itemize}
	\item bezpečnostní složky, agentury
	\item různí správcové (autentizace)
	\item identifikace zločince z kamerového záznamu
	\item možnost najít na fotkách z poslední akce své (mnohdy nové) kamarády
	\item ...
\end{itemize}
\end{frame}

\begin{frame}
\frametitle{Implementační problémy a výhody}

\begin{itemize}
	\item záběry jsou zpravidla pořizovány za různých podmínek
	\item nízká intenzita osvětlení, šum v obrazu, úhel natočení hlavy, ...
	\item nejširší uplatnění především tam, kde je možné kontrolovat podmínky, za kterých jsou tyto obrazy pořizovány
	\item $=>$ biometrická metoda autentizace (na rozdíl od PINů, hesel, ...)
	\item na rozdíl od otisků prstů, sítnice, apod. dvě zásadní výhody: 
		\begin{itemize}
			\item je nejméně invazní
			\item je relativně levná		
		\end{itemize}			
\end{itemize}
\end{frame}

\section{State of the art - identifikace obličeje v obrazu}
\begin{frame}

\begin{center}
\frametitle{State of the art - identifikace obličeje v obrazu}
	\huge State of the art - identifikace obličeje v obrazu\\[1.5cm]
	\large Metody
\end{center}

\end{frame}

\subsection{Metody}
\begin{frame}
\frametitle{Metody}
\begin{itemize}
	\item Vyhledávání jednotlivých částí obličeje
	\item Holistické metody
		\begin{itemize}
			\item Statistické metody
			\item AI metody
		\end{itemize}
\end{itemize}
\end{frame}

\subsection{Vyhledávání jednotlivých částí obličeje}
\begin{frame}
\frametitle{Vyhledávání jednotlivých částí obličeje}
\begin{itemize}
			\item nejstarší metody rozpoznání obličeje
			\item vzorce typické pro dominantní části obličeje
			\item oči, ústa, nos, obočí apod.
			\item měří a počítají jejich vzájemné geometrické vazby
			\item redukují původní obrázek obličeje do vektoru geometrických vlastností (rysů)
			\item statistické metody hledající nejlepší shodu
		\end{itemize}
\end{frame}

\subsection{Holistické metody}
\begin{frame}
\frametitle{Holistické metody}
\begin{itemize}
	\item přistupují k obličeji jako k celku
			\item \textbf{Statistické metody}
				\begin{itemize}
					\item pracují s obrazem ve formě jasové matice
					\item rozpoznání: porovnáním korelací mezi neznámým obličejem a databází známých obličejů
					\item velmi náchylné na podmínky
					\item velmi náročné na výpočetní výkon a paměť
					\item Principal Component Analysis
				\end{itemize}
			\item \textbf{AI metody}
				\begin{itemize}
					\item využívají k identifikaci obličeje neuronové sítě a nástroje strojového učení.
					\item obecně jsou tyto metody úspěšnější
				\end{itemize}
		\end{itemize}
\end{frame}

\section{Eigenfaces, Principal Component Analysis}
\begin{frame}
\begin{center}
\frametitle{Eigenfaces, Principal Component Analysis}
	\huge Eigenfaces, Principal Component Analysis\\[1.5cm]
	\large Co to je?
\end{center}
\end{frame}

%%%%%%%

\subsection{Eigenfaces}
\begin{frame}
\frametitle{Eigenfaces - co to je?}
Ve zkratce:
\begin{itemize}
	\item holistická metoda rozpoznání obličeje v obrazu
	\item vstupní obraz ve formě jasové matice
	\item tréninková sada - převod na pouze charakteristické (rozdílné) znaky každého obličeje
	\item představují model každého obličeje
	\item s ním je porovnáván neznámý obličej určený k identifikaci
	\item =$>$ 2 fáze:
		\begin{itemize}
			\item tréninková
			\item rozpoznávací
		\end{itemize}
\end{itemize}
\end{frame}

\begin{frame}
\frametitle{Eigenfaces - preprocessing}
Než se pustíme do trénování, je potřeba připravit tréninkovou sadu:
\begin{itemize}
			\item oříznout pouze obličej
			\item převést do grayscale (jasová složka)
			\item minimalizovat vel. vstup. prostoru
				\begin{itemize}
					\item resizingem (např. 64x64px)
					\item normalizací intenzity (např. 0-255)
				\end{itemize}							
		\end{itemize}
\end{frame}

\subsection{Tréninková fáze}

\begin{frame}
\frametitle{Eigenfaces - tréninková fáze (1/3)}
\begin{itemize}
\tiny
			\item pomocí \textit{tréninkové sady} - několik fotek ke každé osobě
			\item převod jas. matice obličeje do \textit{obličejového vektoru} (všechny řádky poskládat za sebe - ($m*n$) -$>$ ($(m*n)*1$)) 
			\item sestavit \textit{vektorový prostor obličejů} - poskládáním obličejových vektorů "za sebe" do matice
			\item prostor je silně korelovaný (každý obličej obsahuje oči, nos, ústa, obočí, ...)
			\item =$>$ můžeme vypočítat \textit{průměrný obličej} (společné znaky všech obličejů v trénink. sadě)
			\item odečtení průměrného obličeje od každého obličejového vektoru - zůstanou jen znaky jasně charakterizující (odlišující) daný obličej = \textit{normalizovaný obličejový vektor}
			\end{itemize}

\end{frame}

\begin{frame}
\frametitle{Eigenfaces - tréninková fáze (2/3)}
Redukce dimenze (PCA):
			\begin{itemize}
			\tiny
			\item označme normalizované obličejové vektory jako matici $A$
			\item vypočítáme kovarianční matici: $C=AxA^T$
			\item $C$ obsahuje všechny možné tzv. \textit{eigenvektory} (nebo také \textit{eigenface}) - komponenty jednoho původního obličeje (ten lze zpětně získat lineární kombinací $k$ \textit{eigenvektorů}, kde $k<=$\textit{(počet vstupních obličejů testovací sady)} (+ odečtený \textit{průměrný obličej}))
			\item \textit{eigenvektor} je česky vlastním vektorem kovarianční matice
			\item seřazení vlastních vektorů (\textit{eigenvektor}ů) podle jím příslušejících vlastních čísel = seřazení podle důležitosti (=míry popisnosti původního obličeje)
			\item \textbf{cílem je vybrat $k$ nejvýznamnějších (nejpopisnějších) \textit{eigenvektorů} a zahodit ty méně podstatné =$>$ velmi nepatrné snížení spolehlivosti, ale zásadní redukce dimenze}
		\end{itemize}

\end{frame}

\begin{frame}
\frametitle{Eigenfaces - tréninková fáze (3/3)}
Redukce dimenze (PCA) - efektivnější výpočet \textit{eigenvectorů}:
\begin{itemize}
\tiny
			\item problém je velikost $C$ (obrázek 64x64px má jako vektor rozměr 4096x1, $C$ tak má rozměr 4096x4096 - velká časová náročnost hledání vlastních vektorů)
			\item trik: kovarianční matice s redukovanou dimenzí $C'=A^TxA$ řádu $nxn$ (kde $n$ je počet obličejů ve vstupní \textit{tréninkové sadě})
			\item získání vlastních vektorů z této matice je řádově rychlejší a úspornější
			\item získáme $n$ možných eigenvektorů o rozměru $nx1$ (vs. 4096 eigenvektorů o rozměrech 4096x1)
			\item my ale potřebujeme eigenvektory v původní $C$ s neredukovanou dimenzí, ty získáme mapováním:
				\begin{itemize}
				\tiny
					\item \textit{eigenvektor} v původní dimenzi: $u_i$ 
					\item \textit{eigenvektor} v redukované \textit{kovarianční matici}: $v_i$
					\item mapování: $u_i=Av_i$, kde $A$ je matice\textit{normalizovaných obličejových vektorů}
				\end{itemize}
		\end{itemize}

\end{frame}

%%%%%%%%
\subsection{Rozpoznávací fáze}
\begin{frame}
\frametitle{Rozpoznávací fáze}
Rozpoznání neznámého obličeje pak probíhá následovně:
\begin{itemize}
	\item preprocessing (viz dřívější text)
	\item převedení obličeje do \textit{obličejového vektoru}
	\item normalizace - jako \textit{průměrný obličej} se použije \textit{průměrný obličej} z tréninkové fáze
	\item vypočítat kombinaci $k$ eigenvektorů - projekce na PCA subspace
	\item sestavení\textit{ váhového vektoru} neznámého obličeje (vektor $k$ nejvýznamnějších eigenvectorů)
	\item euklidovské porovnání \textit{váhového vektoru} s \textit{váhovými vektory} získanými v tréninkové fázi (výpočet rozdílu, stanovení hranic pro porovnávání)
\end{itemize}

\end{frame}


\section{Popis implementace}
\begin{frame}

\begin{center}
\frametitle{Popis implementace}
	\huge Popis implementace\\[1.5cm]
	\large Základní use-case
\end{center}

\end{frame}

\subsection{Základní use-case}
\begin{frame}
\frametitle{Základní use-case}
\begin{itemize}
	\item Mám k dispozici sadu fotografií známých osob. Tyto nahraji do systému (proběhne tréninková fáze). Dále předložím systému novou, neznámou fotografii a budu chtít identifikovat osobu, která se na fotografii nachází.
	\item Vstupní data:
		\begin{itemize}
			\item Některá z veřejných databází (AT\&T)
			\item Fotografie by měly být pořízeny za stejných světelných podmínek, stejnou technikou, ze stejné vzdálenosti a se stejným výrazem pro všechny fotografované osoby
		\end{itemize}
\end{itemize}
\end{frame}



\subsection{Použité technologie}
\begin{frame}
\frametitle{Použité technologie}

\begin{itemize}
	\item systém naprogramován v Javě 1.7
	\item pro perzistenci výsledků tréninkové fáze použita databáze H2 a ORM/JPA provider Hibernate
	\item pro práci s obrázky použita knihovna OpenCV a oficiální Java wrapper.
	\item kompatibilní s OS Windows a Linux
\end{itemize}

Dále pro sestavení aplikace je potřeba:

\begin{itemize}
		\item Git
		\item Java Development Kit 1.7
		\item Maven $>$=2
		\item pdflatex pro případne sestavení doprovodného dokumentu a prezentace
\end{itemize}

\end{frame}

\section{Použití aplikace}
\begin{frame}[fragile]
\frametitle{Použití aplikace}
\tiny
\begin{verbatim}
usage: uber-Eigenfaces-1.0-SNAPSHOT.jar
 -af,--save-average-face <arg>          Saves the average face to a
                                        specified file.
 -c,--cleanup-database                  clean up database containing the
                                        last training set
 -d,--debug                             enables the debug mode (maximal
                                        verbosity)
 -h,--help                              show this help
 -l,--log <arg>                         enables the logging to a file
 -p,--persist-training-set              Persists the provided training set
 -r,--recognize-face <arg>              path to the image file required to
                                        recognize
 -t,--training-set-xml <arg>            path to the XML containing the
                                        training set
 -te,--test-recognizer-existing-faces   tests the recognizer's confidence
                                        by attempting to recognize each
                                        item in the training set
 -ts,--test-recognizer-skipped-faces    tests the recognizer's confidence
                                        by attempting to recognize one
                                        skipped face of each person in the
                                        training set
                                        
\end{verbatim}
\normalsize
\end{frame}

\section{Závěr a výsledky}
\begin{frame}
\frametitle{Závěr a výsledky}
Aplikace obsahuje dva vestavěné testovací scénáře:

\begin{itemize}
	\item První z nich (parametr \textit{-te}) prochází každý obrázek v tréninkové sadě a snaží se jej rozpoznat. Je očekávána 100\% úspěšnost s maximální jistotou rozpoznání, což aplikace splňuje.
	\item Druhý scénář (parametr \textit{-ts}) vyžaduje, aby ke každé osobě existovaly v tréninkové sadě alespoň dvě fotografie. Následně je od každé osoby jedna fotografie vyjmuta, aplikace je znovu "přetrénována" s touto redukovanou tréninkovou sadou a vyjmuté fotografie jsou použity jako testovací (k rozpoznání). Dosažená úspěšnost je u vzorové testovací sady 97,5\%, což odpovídá výsledkům dosahovaných u eigenfaces algoritmu.
\end{itemize}
\end{frame}




%%%%%%%%%%%%%%%%%%%%% PODEKOVANI %%%%%%%%%%%%%%%%

\begin{frame}
\begin{center}
	\large \textit{... děkuji za pozornost ...}
\end{center}
\end{frame}


\end{document}